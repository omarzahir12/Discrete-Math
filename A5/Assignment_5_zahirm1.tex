\documentclass[11pt,fleqn]{article}

\setlength {\topmargin} {-.15in}
\setlength {\textheight} {8.6in}

\usepackage{ulem}

\usepackage{amsmath}
\usepackage{amssymb}
\usepackage{amsthm}
\usepackage{url}
\usepackage{color}
\usepackage{tikz}
\usetikzlibrary{automata,positioning,arrows}

\setlength {\topmargin} {-.15in}
\setlength {\textheight} {8.6in}

\renewcommand{\labelenumi}{\theenumi.}
\renewcommand{\labelenumii}{\theenumii.}
\renewcommand{\labelenumiii}{\theenumiii.}
\newcommand{\be}{\begin{enumerate}}
	\newcommand{\ee}{\end{enumerate}}
\newcommand{\bi}{\begin{itemize}}
	\newcommand{\ei}{\end{itemize}}
\newcommand{\bc}{\begin{center}}
	\newcommand{\ec}{\end{center}}
\newcommand{\bsp}{\begin{sloppypar}}
	\newcommand{\esp}{\end{sloppypar}}
\newcommand{\mname}[1]{\mbox{\sf #1}}
\newcommand{\sB}{\mbox{$\cal B$}}
\newcommand{\sC}{\mbox{$\cal C$}}
\newcommand{\sF}{\mbox{$\cal F$}}
\newcommand{\sM}{\mbox{$\cal M$}}
\newcommand{\sP}{\mbox{$\cal P$}}
\newcommand{\sV}{\mbox{$\cal V$}}
\newcommand{\set}[1]{{\{ #1 \}}}
\newcommand{\Neg}{\neg}
\ifdefined \And
\renewcommand{\And}{\wedge}
\else
\newcommand{\And}{\wedge}
\fi
\newcommand{\Or}{\vee}
\newcommand{\Implies}{\Rightarrow}
\newcommand{\Iff}{\LeftRightarrow}
\newcommand{\Forall}{\forall}
\newcommand{\ForallApp}{\forall\,}
\newcommand{\Forsome}{\exists}
\newcommand{\ForsomeApp}{\exists\,}
\newcommand{\mdot}{\mathrel.}
\newcommand{\eps}{\epsilon}
\newcommand{\pnote}[1]{\langle \mbox{#1} \rangle}


\begin{document}
	
	\begin{center}
		
		{\large \textbf{COMPSCI/SFWRENG 2FA3}}\\[2mm]
		{\large \textbf{Discrete Mathematics with Applications II}}\\[2mm]
		{\large \textbf{Winter 2021}}\\[8mm]
		{\huge \textbf{Assignment 5}}\\[6mm]
		{\large \textbf{Dr.~William M. Farmer and Dr.~Mehrnoosh Askarpour}}\\[2mm]
		{\large \textbf{McMaster University}}\\[6mm]
		{\large Revised: March 2, 2021}
		
	\end{center}
	
	\medskip
	
	Assignment 5 consists of a problem with three subproblems.  You must write your solutions
	to the problems using LaTeX.
	
	Please submit Assignment~5 as two files,
	\texttt{Assignment\_5\_\emph{YourMacID}.tex} and
	\texttt{Assignment\_5\_\emph{YourMacID}.pdf}, to the Assignment~5
	folder on Avenue under Assessments/Assignments.
	\texttt{\emph{YourMacID}} must be your personal MacID (written without
	capitalization).  The \texttt{Assignment\_5\_\emph{YourMacID}.tex}
	file is a copy of the LaTeX source file for this assignment
	(\texttt{Assignment\_5.tex} found on Avenue under
	Contents/Assignments) with your solution entered after each problem.
	The \texttt{Assignment\_5\_\emph{YourMacID}.pdf} is the PDF output
	produced by executing
	
	\begin{itemize}
		
		\item[] \texttt{pdflatex Assignment\_5\_\emph{YourMacID}}
		
	\end{itemize}
	
	This assignment is due \textbf{Sunday, March 7, 2021 before
		midnight.}  You are allow to submit the assignment multiple times,
	but only the last submission will be marked.  \textbf{Late submissions
		and files that are not named exactly as specified above will not be
		accepted!}  It is suggested that you submit your preliminary
	\texttt{Assignment\_5\_\emph{YourMacID}.tex} and
	\texttt{Assignment\_5\_\emph{YourMacID}.pdf} files well before the
	deadline so that your mark is not zero if, e.g., your computer fails
	at 11:50 PM on March 7.
	
	\textbf{Although you are allowed to receive help from the
		instructional staff and other students, your submission must be your
		own work.  Copying will be treated as academic dishonesty! If any of
		the ideas used in your submission were obtained from other students
		or sources outside of the lectures and tutorials, you must
		acknowledge where or from whom these ideas were obtained.}
	
	\newpage
	
	\subsection*{Presenting DFAs and NFAs Transition Diagrams}
	
	In this assignment you are asked to present DFAs as transition
	diagrams.  You are can do this in one of two ways.
	
	The first way is to present the diagram using the LaTeX graphics
	package TikZ.  The TikZ code can either be written by hand or
	automatically generated using the finsm system available at
	\texttt{http:finsm.io}.
	
	Here are some examples of how it can be used to create
	DFA and NFA transition diagrams that appear in the lectures slides:
	
	\begin{center}
		\begin{tikzpicture}[shorten >=1pt,node distance=2.5cm,on grid,auto] 
			\node[state, initial, thick] (q_0)   {$q_0$}; 
			\node[state, thick] (q_1) [right=of q_0] {$q_1$}; 
			\node[state, thick] (q_2) [right=of q_1] {$q_2$}; 
			\node[state, accepting, thick] (q_3) [right=of q_2] {$q_3$};
			\path[->, thick, >=stealth] 
			(q_0) edge [bend left, above] node {$a$} (q_1)
			edge [loop, above] node {$b$} (q_2)
			(q_1) edge [bend left, above] node {$a$} (q_2)
			edge [bend left, above] node {$b$} (q_0)
			(q_2) edge [bend left, above] node {$a$} (q_3) 
			edge [bend left, below]  node {$b$} (q_0)
			(q_3) edge [loop, above] node {$a,b$} (q_2); 
		\end{tikzpicture}
	\end{center}
	
	\begin{center}
		\begin{tikzpicture}[shorten >=1pt,node distance=4cm,on grid,auto] 
			\node[state, initial, accepting, thick] (q_0)   {$q_0$}; 
			\node[state, thick] (q_1) [right=of q_0] {$q_1$}; 
			\node[state, thick] (q_2) [below=of q_0] {$q_2$}; 
			\node[state, thick] (q_3) [right=of q_2] {$q_3$};
			\path[->, thick, >=stealth] 
			(q_0) edge [bend left, above] node {1} (q_1)
			edge [bend left, right] node {0} (q_2)
			(q_1) edge [bend left, below] node {1} (q_0)
			edge [bend left, right] node {0} (q_3)
			(q_2) edge [bend left, above] node {1} (q_3) 
			edge [bend left, left]  node {0} (q_0)
			(q_3) edge [bend left, below] node {1} (q_2) 
			edge [bend left, left]  node {0} (q_1);
		\end{tikzpicture}
	\end{center}
	
	\begin{center}
		\begin{tikzpicture}[shorten >=1pt,node distance=1.7cm,on grid,auto] 
			\node[state, initial, thick] (q_0)   {$q_0$}; 
			\node[state, thick] (q_1) [right=of q_0] {$q_1$}; 
			\node[state, thick] (q_2) [right=of q_1] {$q_2$}; 
			\node[state, thick] (q_3) [right=of q_2] {$q_3$};
			\node[state, thick] (q_4) [right=of q_3] {$q_4$};
			\node[state, thick, accepting] (q_5) [right=of q_4] {$q_5$};
			\path[->, thick, >=stealth] 
			(q_0) edge [loop, above] node {0,1} (q_1)
			edge [right, above] node {1} (q_1)
			(q_1) edge [right, above] node {0,1} (q_2)
			(q_2) edge [right, above] node {0,1} (q_3)
			(q_3) edge [right, above] node {0,1} (q_4)
			(q_4) edge [right, above] node {0,1} (q_5);
		\end{tikzpicture}
	\end{center}
	
	The second way is to take a picture of a hand-written transition
	diagram and then embed it into your assignment using the following
	LaTeX code:
	\begin{verbatim}
		\begin{center}
			\includegraphics[scale = 0.5]{diagram.jpg}
		\end{center}
	\end{verbatim}
	Please make sure your diagram is legible.
	
	\subsection*{Problems}
	Let $\Sigma = \{a, b, c\}$ and $L =\{x \in
        \Sigma^* \mid x \text{ contains ``cc"}\}$: \be
	
	\item \textbf{[7 points]} Construct an NFA that accepts L.
	
	\item \textbf{[8 points]} Construct the equivalent DFA using the subset construction.
	
	\item \textbf{[5 points]} Is the constructed DFA minimal? If not, construct the equivalent minimal DFA.

        \ee

        \noindent
        Present your FAs using transition diagrams.
        
        \bigskip

        \noindent
        \textcolor{blue}{\textbf{Mohammad Omar Zahir, zahirm1, March 7th }}
    
    \medskip
    \begin{enumerate}
     \item Below is the constructed NFA that accepts L.
    \begin{center}
        \begin{tikzpicture}[]
                \node[initial,thick,state] at (-4.7,1.125) (f3c0d843) {$q_0$};
                \node[thick,state] at (-2.325,1.125) (93d387af) {$q_1$};
                \node[thick,accepting,state] at (0.1,1.1) (94a27a87) {$q_2$};
                \path[->, thick, >=stealth]
                (f3c0d843) edge [loop,min distance = 1.25cm,above,in = 121, out = 59] node {$a,b,c$} (f3c0d843)
                (f3c0d843) edge [below,in = 180, out = 0] node {$c$} (93d387af)
                (93d387af) edge [below,in = 179, out = -1] node {$c$} (94a27a87)
                (94a27a87) edge [loop,min distance = 1.25cm,above,in = 121, out = 59] node {$a,b,c$} (94a27a87)
                ;
        \end{tikzpicture}
    \end{center}
    
    \medskip
    \item The following is the equivalent DFA, along with the subset construction table for the above NFA. The crossed out sections are inaccessible, and do not have to be included. Since $q_0$ is the single start state in the NFA, it is also the start state for our equivalent DFA. Similar to the start state, because $q_2$ was the end state in the NFA, and it is used in $q_{012}$ and $q_{02}$, it makes these states the end states.
    \begin{center}
        \begin{tabular}{|p{1cm}|p{1cm}|p{1cm}|p{1cm}|}
         \hline
         \multicolumn{4}{|c|}{$\Sigma$} \\
         \hline
         $P(Q)$ & a & b & c\\
         \hline
         \textcolor{black}{\sout{$\emptyset$}} & \textcolor{black}{\sout{$\emptyset$}} & \textcolor{black}{\sout{$\emptyset$}} & \textcolor{black}{\sout{$\emptyset$}}\\
         
         \textcolor{black}{$q_0$} & \textcolor{black}{$q_0$} &
         \textcolor{black}{$q_0$} & \textcolor{black}{$q_{01}$}\\
         
         \textcolor{black}{\sout{$q_1$}} & \textcolor{black}{\sout{$\emptyset$}} &
         \textcolor{black}{\sout{$\emptyset$}} & \textcolor{black}{\sout{$q_2$}}\\
         
         \textcolor{black}{\sout{$q_2$}} & \textcolor{black}{\sout{$q_2$}} &
         \textcolor{black}{\sout{$q_2$}} & \textcolor{black}{\sout{$q_2$}}\\
         
         \textcolor{black}{$q_{01}$} & \textcolor{black}{$q_0$} &
         \textcolor{black}{$q_0$} & \textcolor{black}{$q_{012}$}\\ 
         
         \textcolor{black}{$q_{02}$} & \textcolor{black}{$q_{02}$} &
         \textcolor{black}{$q_{02}$} & \textcolor{black}{$q_{012}$}\\
         
         \textcolor{black}{\sout{$q_{12}$}} & \textcolor{black}{\sout{$q_2$}} &
         \textcolor{black}{\sout{$q_2$}} & \textcolor{black}{\sout{$q_2$}}\\
         
         \textcolor{black}{$q_{012}$} & \textcolor{black}{$q_{02}$} &
         \textcolor{black}{$q_{02}$} & \textcolor{black}{$q_{012}$}\\
         \hline
    	\end{tabular}
    	
    \begin{tikzpicture}[]
        \node[initial,thick,state] at (-3.5,0.5) (6e1d8ae1) {$q_0$};
        \node[thick,state] at (-0.5,0.5) (540bdffb) {$q_{01}$};
        \node[thick,accepting,state] at (2.5,0.5) (5c4c3705) {$q_{012}$};
        \node[thick,accepting,state] at (5.625,0.475) (af4834d3) {$q_{02}$};
        \path[->, thick, >=stealth]
        (6e1d8ae1) edge [above,in = 171, out = 9] node {$c$} (540bdffb)
        (6e1d8ae1) edge [loop,min distance = 1.25cm,above,in = 121, out = 59] node {$a,b$} (6e1d8ae1)
        (540bdffb) edge [below,in = -9, out = -171] node {$a,b$} (6e1d8ae1)
        (540bdffb) edge [below,in = 180, out = 0] node {$c$} (5c4c3705)
        (5c4c3705) edge [above,in = 170, out = 9] node {$a,b$} (af4834d3)
        (5c4c3705) edge [loop,min distance = 1.25cm,above,in = 121, out = 59] node {$c$} (5c4c3705)
        (af4834d3) edge [below,in = -10, out = -171] node {$c$} (5c4c3705)
        (af4834d3) edge [loop,min distance = 1.26cm,above,in = 120, out = 58] node {$a,b$} (af4834d3)
        ;
    \end{tikzpicture}
    \end{center}
    \medskip
    \newpage
    \item
    Despite the fact that the DFA constructed above already excludes the inaccessible states that have been labelled, it is still not minimal. The definition of a minimal DFA would be that there are no unnecessary states, which is not the case since we can see that the transition from $q_{012}$ to $q_{02}$ can be simplified to just simply $q_{012}$, as the transitions within $q_{02}$ are essentially repeated, and can be moved to the single state $q_{012}$, as is done below.
    \begin{center}
        \begin{tikzpicture}[]
            \node[initial,thick,state] at (-3.475,0.925) (317bb5e2) {$q_0$};
            \node[thick,state] at (-0.325,0.9) (7577128f) {$q_{01}$};
            \node[thick,accepting,state] at (2.725,0.85) (a5d4780e) {$q_{012}$};
            \path[->, thick, >=stealth]
            (317bb5e2) edge [above,in = 171, out = 9] node {$c$} (7577128f)
            (317bb5e2) edge [loop,min distance = 1.25cm,above,in = 121, out = 59] node {$a,b$} (317bb5e2)
            (7577128f) edge [below,in = -9, out = -171] node {$a,b$} (317bb5e2)
            (7577128f) edge [above,in = 178, out = 0] node {$c$} (a5d4780e)
            (a5d4780e) edge [loop,min distance = 1.25cm,above,in = 121, out = 59] node {$a,b,c$} (a5d4780e)
            ;
        \end{tikzpicture}
    \end{center}
    \end{enumerate}

        \noindent
        
        
\end{document}


