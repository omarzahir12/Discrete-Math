\documentclass[11pt,fleqn]{article}

\setlength {\topmargin} {-.15in}
\setlength {\textheight} {8.6in}

\usepackage{amsmath}
\usepackage{amssymb}
\usepackage{amsthm}
\usepackage{url}
\usepackage{color}
\usepackage{tikz}
\usetikzlibrary{automata,positioning,arrows}

\setlength {\topmargin} {-.15in}
\setlength {\textheight} {8.6in}

\renewcommand{\labelenumi}{\theenumi.}
\renewcommand{\labelenumii}{\theenumii.}
\renewcommand{\labelenumiii}{\theenumiii.}
\newcommand{\be}{\begin{enumerate}}
	\newcommand{\ee}{\end{enumerate}}
\newcommand{\bi}{\begin{itemize}}
	\newcommand{\ei}{\end{itemize}}
\newcommand{\bc}{\begin{center}}
	\newcommand{\ec}{\end{center}}
\newcommand{\bsp}{\begin{sloppypar}}
	\newcommand{\esp}{\end{sloppypar}}
\newcommand{\mname}[1]{\mbox{\sf #1}}
\newcommand{\sB}{\mbox{$\cal B$}}
\newcommand{\sC}{\mbox{$\cal C$}}
\newcommand{\sF}{\mbox{$\cal F$}}
\newcommand{\sM}{\mbox{$\cal M$}}
\newcommand{\sP}{\mbox{$\cal P$}}
\newcommand{\sV}{\mbox{$\cal V$}}
\newcommand{\set}[1]{{\{ #1 \}}}
\newcommand{\Neg}{\neg}
\ifdefined \And
\renewcommand{\And}{\wedge}
\else
\newcommand{\And}{\wedge}
\fi
\newcommand{\Or}{\vee}
\newcommand{\Implies}{\Rightarrow}
\newcommand{\Iff}{\LeftRightarrow}
\newcommand{\Forall}{\forall}
\newcommand{\ForallApp}{\forall\,}
\newcommand{\Forsome}{\exists}
\newcommand{\ForsomeApp}{\exists\,}
\newcommand{\mdot}{\mathrel.}
\newcommand{\eps}{\epsilon}
\newcommand{\pnote}[1]{\langle \mbox{#1} \rangle}


\begin{document}

\begin{center}

{\large \textbf{COMPSCI/SFWRENG 2FA3}}\\[2mm]
{\large \textbf{Discrete Mathematics with Applications II}}\\[2mm]
{\large \textbf{Winter 2021}}\\[8mm]
{\huge \textbf{Assignment 4}}\\[6mm]
{\large \textbf{Dr.~William M. Farmer and Dr.~Mehrnoosh Askarpour}}\\[2mm]
{\large \textbf{McMaster University}}\\[6mm]
{\large Revised: February 23, 2021}

\end{center}

\medskip

Assignment 4 consists of two problems.  You must write your solutions
to the problems using LaTeX.

Please submit Assignment~4 as two files,
\texttt{Assignment\_4\_\emph{YourMacID}.tex} and
\texttt{Assignment\_4\_\emph{YourMacID}.pdf}, to the Assignment~4
folder on Avenue under Assessments/Assignments.
\texttt{\emph{YourMacID}} must be your personal MacID (written without
capitalization).  The \texttt{Assignment\_4\_\emph{YourMacID}.tex}
file is a copy of the LaTeX source file for this assignment
(\texttt{Assignment\_4.tex} found on Avenue under
Contents/Assignments) with your solution entered after each problem.
The \texttt{Assignment\_4\_\emph{YourMacID}.pdf} is the PDF output
produced by executing

\begin{itemize}

  \item[] \texttt{pdflatex Assignment\_4\_\emph{YourMacID}}

\end{itemize}

This assignment is due \textbf{Sunday, February 28, 2021 before
  midnight.}  You are allow to submit the assignment multiple times,
but only the last submission will be marked.  \textbf{Late submissions
  and files that are not named exactly as specified above will not be
  accepted!}  It is suggested that you submit your preliminary
\texttt{Assignment\_4\_\emph{YourMacID}.tex} and
\texttt{Assignment\_4\_\emph{YourMacID}.pdf} files well before the
deadline so that your mark is not zero if, e.g., your computer fails
at 11:50 PM on February 28.

\textbf{Although you are allowed to receive help from the
  instructional staff and other students, your submission must be your
  own work.  Copying will be treated as academic dishonesty! If any of
  the ideas used in your submission were obtained from other students
  or sources outside of the lectures and tutorials, you must
  acknowledge where or from whom these ideas were obtained.}

\newpage

\subsection*{Presenting DFAs and NFAs Transition Diagrams}

In this assignment you are asked to present DFAs as transition
diagrams.  You are can do this in one of two ways.

The first way is to present the diagram using the LaTeX graphics
package TikZ.  The TikZ code can either be written by hand or
automatically generated using the finsm system available at
\texttt{http:finsm.io}.

Here are some examples of how it can be used to create
DFA and NFA transition diagrams that appear in the lectures slides:

\begin{center}
\begin{tikzpicture}[shorten >=1pt,node distance=2.5cm,on grid,auto] 
   \node[state, initial, thick] (q_0)   {$q_0$}; 
   \node[state, thick] (q_1) [right=of q_0] {$q_1$}; 
   \node[state, thick] (q_2) [right=of q_1] {$q_2$}; 
   \node[state, accepting, thick] (q_3) [right=of q_2] {$q_3$};
    \path[->, thick, >=stealth] 
    (q_0) edge [bend left, above] node {$a$} (q_1)
          edge [loop, above] node {$b$} (q_2)
    (q_1) edge [bend left, above] node {$a$} (q_2)
          edge [bend left, above] node {$b$} (q_0)
    (q_2) edge [bend left, above] node {$a$} (q_3) 
          edge [bend left, below]  node {$b$} (q_0)
    (q_3) edge [loop, above] node {$a,b$} (q_2); 
\end{tikzpicture}
\end{center}

\begin{center}
\begin{tikzpicture}[shorten >=1pt,node distance=4cm,on grid,auto] 
   \node[state, initial, accepting, thick] (q_0)   {$q_0$}; 
   \node[state, thick] (q_1) [right=of q_0] {$q_1$}; 
   \node[state, thick] (q_2) [below=of q_0] {$q_2$}; 
   \node[state, thick] (q_3) [right=of q_2] {$q_3$};
    \path[->, thick, >=stealth] 
    (q_0) edge [bend left, above] node {1} (q_1)
          edge [bend left, right] node {0} (q_2)
    (q_1) edge [bend left, below] node {1} (q_0)
          edge [bend left, right] node {0} (q_3)
    (q_2) edge [bend left, above] node {1} (q_3) 
          edge [bend left, left]  node {0} (q_0)
    (q_3) edge [bend left, below] node {1} (q_2) 
          edge [bend left, left]  node {0} (q_1);
\end{tikzpicture}
\end{center}

\begin{center}
\begin{tikzpicture}[shorten >=1pt,node distance=1.7cm,on grid,auto] 
   \node[state, initial, thick] (q_0)   {$q_0$}; 
   \node[state, thick] (q_1) [right=of q_0] {$q_1$}; 
   \node[state, thick] (q_2) [right=of q_1] {$q_2$}; 
   \node[state, thick] (q_3) [right=of q_2] {$q_3$};
   \node[state, thick] (q_4) [right=of q_3] {$q_4$};
   \node[state, thick, accepting] (q_5) [right=of q_4] {$q_5$};
    \path[->, thick, >=stealth] 
    (q_0) edge [loop, above] node {0,1} (q_1)
          edge [right, above] node {1} (q_1)
    (q_1) edge [right, above] node {0,1} (q_2)
    (q_2) edge [right, above] node {0,1} (q_3)
    (q_3) edge [right, above] node {0,1} (q_4)
    (q_4) edge [right, above] node {0,1} (q_5);
\end{tikzpicture}
\end{center}

The second way is to take a picture of a hand-written transition
diagram and then embed it into your assignment using the following
LaTeX code:
\begin{verbatim}
\begin{center}
\includegraphics[scale = 0.5]{diagram.jpg}
\end{center}
\end{verbatim}
Please make sure your diagram is legible.

\subsection*{Problems}

\be

  \item \textbf{[10 points]} Construct a deterministic finite
    automaton (DFA) $A$ for the alphabet $\Sigma = \set{0,1}$ such
    that $L(A)$ is the set of all strings $x$ in $\Sigma^*$ for which
    $\#0(x)$ is even and $\#1(x)$ is divisible by 3.  Present $A$ as a
    transition diagram.

  \textcolor{blue}{\textbf{Mohammad Omar Zahir, zahirm1, Feb 28, 2021}}\\
    \begin{center}
    \begin{tikzpicture}[]
        \node[initial,thick,accepting,state] at (-1.925,2.075) (19539728) {$q_0$};
        \node[thick,state] at (1.45,2.125) (9413e3b5) {$q_1$};
        \node[thick,state] at (-1.9,-1.475) (299c8301) {$q_3$};
        \node[thick,state] at (1.45,-1.525) (a97a0ad7) {$q_4$};
        \node[thick,state] at (4.6,2.0625) (e0402122) {$q_{2}$};
        \node[thick,state] at (4.625,-1.4625) (e077fb62) {$q_{5}$};
        \path[->, thick, >=stealth]
        (19539728) edge [above,in = 173, out = 9] node {$1$} (9413e3b5)
        (19539728) edge [right,in = 82, out = -82] node {$0$} (299c8301)
        (9413e3b5) edge [right,in = 82, out = -82] node {$0$} (a97a0ad7)
        (9413e3b5) edge [above,in = 169, out = 8] node {$1$} (e0402122)
        (299c8301) edge [above,in = 171, out = 8] node {$1$} (a97a0ad7)
        (299c8301) edge [left,in = -98, out = 98] node {$0$} (19539728)
        (a97a0ad7) edge [left,in = -98, out = 98] node {$0$} (9413e3b5)
        (a97a0ad7) edge [above,in = 171, out = 11] node {$1$} (e077fb62)
        (e0402122) edge [left,in = 104, out = -103] node {$0$} (e077fb62)
        (e0402122) edge [above,in = 34, out = 144] node {$1$} (19539728)
        (e077fb62) edge [right,in = -78, out = 79] node {$0$} (e0402122)
        (e077fb62) edge [below,in = -28, out = -149] node {$1$} (299c8301)
        ;
    \end{tikzpicture}
    \end{center}

  \bigskip

  \item \textbf{[10 points]} Let $B$ be the DFA given by the following
    transition diagram:

	\begin{center}
	\begin{tikzpicture}[]
		\node[thick,initial,state] (q0) {$q_{0}$};
		\node[thick,accepting, state] (q1)  [right= of q0]{$q_{1}$};
		\node[thick,state] (q2)  [right= of q1]{$q_{2}$};
		\path[->, thick, >=stealth]
		(q0) edge [bend left,above] node {$1$} (q1)
		edge [loop above]   node         {0} ()
		(q1) edge [bend left,above] node {$0$} (q0)
		edge [bend left,above] node {$1$} (q2)
		(q2) edge [bend left,above] node {$1$} (q1)
		edge [bend left = 40, below] node {$0$} (q0)
		;
	\end{tikzpicture}
        \end{center}

   Prove that $L(B)$ is the language of all binary strings that end
   with an odd number of $1$s.  Hint: Use weak induction on the length
   of the input string, and let $P(n)$ be the statement that, for all
   input strings $w$ with $|w| = n$, the following conditions hold:

   \be

     \item If $\delta^*(q_0,w) = q_0$, then $w$ is $\epsilon$ or ends
       with $0$.

     \item If $\delta^*(q_0,w) = q_1$, then $w$ ends with an odd
       number of $1$s.

     \item If $\delta^*(q_0,w) = q_2$, then $w$ ends
       with an even number of $1$s.

   \ee
	
   \textcolor{blue}{\textbf{Mohammad Omar Zahir, zahirm1, Feb 28, 2021}}\\
   \begin{proof}
   We will prove that $L(B)$ is the language of all binary strings that end
   with an odd number of $1$s. We will prove $P(n)$ for the the three cases $P_1$, $P_2$, $P_3$, corresponding to a, b, c in the question respectively, where $n = |w|$ the length of the word, by weak induction. \\
   
   \emph{Base Case: n = 1}\\
   We will first prove $P_1$, $P_2$, $P_3$ for the base case, where $n = |w| = 0$, or the empty word $\eps$, and prove $P(0)$.\\\\ 
   \emph{P1:}
   We know this to be trivially true as this is what is stated in the condition for $w = \eps$, which is the empty word. Thus it holds for $P1$.
   \\\\$\delta^*(q_0,\eps) = q_0$\\\\
   \emph{P2:}
   We know this to be true because the condition states that when the state is at $q_1$, the word must end with an odd number of $1$s. Because we have the empty word $|w| = 0$, that means that our word does not end with anything, meaning our state never leaves $q_0$. This can be modelled as an implication like $\delta^*(q_0,w) = q_1 \implies w$. We know that the left side of our implication must be False since we stay at $q_0$ for the empty word, meaning that we get $False \implies w$ which is equivalent to $True$ by `Ex-Falso Quodlibet'. Thus $P_2$ holds.\\\\
   \emph{P3:}
   This is proved identically to $P_2$. We know this to be true because the condition states that when the state is at $q_2$, the word must end with an even number of $1$s. Because we have the empty word $|w| = 0$, that means that our word does not end with anything, meaning our state never leaves $q_0$. This can be modelled as an implication like $\delta^*(q_0,w) = q_2 \implies w$. We know that the left side of our implication must be False since we stay at $q_0$ for the empty word, meaning that we get $False \implies w$ which is equivalent to $True$ by `Ex-Falso Quodlibet'. Thus $P_3$ holds.\\
   Having proved the three base case properties, we can say P(0) holds.\\\\
   \emph{Induction Step: n $>$ 0}\\
   With the single base cases proven for each $P_1$, $P_2$, $P_3$, we will thus prove $P(n+1)$ holds for all $n = |w| > 0$. Specifically, $w = xi$ which represents the concatenation of $x$ and $i$ where $x \in \Sigma^*$ and $i$ is the end of the word specified by the transition to the state. Additionally, $|w| = n + 1$ and $|x| = n$, and as we are using weak induction, we assume $P(n)$ holds and prove for $P(n+1)$. \\\\
   \emph{P1:}
   To prove this, we must show that $w$ must end with 0, as $w$ can no longer be $\eps$. Since we know that the we must end in the $q_0$ state for this property, we can analyse the transitions that are followed to get to $q_0$. 
   \begin{align*}
      &\phantom{{}=} \mname {$\delta^*(q_0, w)$}\\
      &= \mname{$\delta^*(q_0, xi)$} & \pnote{Substitution}\\
      &= \mname{$\delta(\delta^*(q_0, x), i)$} & \pnote{Definition of $\delta^*$}\\
      &= \mname{$q_0$} & \pnote{R.H.S. of $P_1$}
   \end{align*}
   We can see that all the transitions, represented by the arrows, that are pointing into or ending at $q_0$ are ending with a 0, meaning that if the state is at $q_0$ the last number of the word must be 0. This can formally be written as $\delta^*(q, 0)$ for all $q \in Q$, where $Q$ is all possible states. Thus $P1$ holds, as we have satisfied the condition for all cases for each state.
   \\\\
   \emph{P2:}
   To prove this, we can model the situation as follows.
   \begin{align*}
      &\phantom{{}=} \mname {$\delta^*(q_0, w)$}\\
      &= \mname{$\delta^*(q_0, xi)$}  & \pnote{Substitution}\\
      &= \mname{$\delta(\delta^*(q_0, x), i)$}  & \pnote{Definition of $\delta^*$}\\
      &= \mname{$q_1$} & \pnote{R.H.S. of $P_2$ (For Induction Step)}
   \end{align*}
   Here we know that $i$ must represent $1$ for both cases as both the transitions to $q_1$ end with a 1. Therefore to prove this we have to analyse the possibilities of the two cases for $q_1$.\\\\
   \emph{Case 1: $\delta(q_0, 1)$}\\
   Here we can see that $q_0$ represents $\delta^*(q_0, x) = q_0$. Since we are using weak induction and assuming $P(n)$, we can apply our induction hypothesis for $P_1$ on $x$ which has length of $n$, giving us an empty word or a word that ends with 0. We will then concatenate 1, as we know that $i = 1$, and in either case, we will end the word with a single $1$, which is an odd number of $1$s. Thus, we have proved that $Case$ 1 holds.\\\\
   \emph{Case 2: $\delta(q_2, 1)$}\\
   Here we can see that $q_2$ represents $\delta^*(q_0, x) = q_2$. Since we are using weak induction and assuming $P(n)$, we can apply our induction hypothesis for $P_3$ on $x$ which has length of $n$, giving us a word that ends with an even number of $1$s. We will then concatenate 1, as we know that $i = 1$, and the word will end with adding a single $1$, which is an odd number of $1$s, after an even number of $1$s, which is an odd number of $1$s since we trivially know that even + odd = odd. Thus, we have proved that $Case$ 2 holds.\\\\
   Having proven that the two distinct Cases for $P_2$ hold, we can say that $P_2$ holds.
   \\\\
   \emph{P3:}
   To prove this, we can model the situation as follows.
   \begin{align*}
      &\phantom{{}=} \mname {$\delta^*(q_0, w)$} \\
      &= \mname{$\delta^*(q_0, xi)$} & \pnote{Substitution}\\ 
      &= \mname{$\delta(\delta^*(q_0, x), i)$} & \pnote{Definition of $\delta^*$}\\
      &= \mname{$q_2$} & \pnote{R.H.S. of $P_3$ (For Induction Step)}
   \end{align*}
   Here we know that $i$ must represent $1$ for the single transition case for the $q_2$ state which can be proved as a case below.\\\\
   \emph{Case 1: $\delta(q_1, 1)$}\\
   Here we can see that $q_1$ represents $\delta^*(q_0, x) = q_1$. Since we are using weak induction and assuming $P(n)$, we can apply our induction hypothesis for $P_2$ on $x$ which has length of $n$. This gives us a word that ends with an odd number of $1$s based on the induction hypothesis. We will then concatenate 1, as we know that $i = 1$ from the transition, which is also an odd number of $1$s. Their concatenation will give us an even number of $1$s as we trivially know that odd + odd = even. Thus, we have proved that $Case$ 1 holds.\\\\
   Having proven the single distinct case for $P_3$, we can say that $P_3$ holds.\\
   
    We have proven $P(n+1)$ holds for the three cases $P_1$, $P_2$, $P_3$, where $n = |w|$ the length of the word. Thus we have proved that $P(n)$ holds by weak induction.
   
   \end{proof}
\ee

\end{document}
