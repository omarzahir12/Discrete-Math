\documentclass[11pt,fleqn]{article}

\setlength {\topmargin} {-.15in}
\setlength {\textheight} {8.6in}

\usepackage{amsmath}
\usepackage{amssymb}
\usepackage{amsthm}
\usepackage{url}
\usepackage{color}
\usepackage{tikz}
\usetikzlibrary{automata,positioning,arrows}

\setlength {\topmargin} {-.15in}
\setlength {\textheight} {8.6in}

\renewcommand{\labelenumi}{\theenumi.}
\renewcommand{\labelenumii}{\theenumii.}
\renewcommand{\labelenumiii}{\theenumiii.}
\newcommand{\be}{\begin{enumerate}}
	\newcommand{\ee}{\end{enumerate}}
\newcommand{\bi}{\begin{itemize}}
	\newcommand{\ei}{\end{itemize}}
\newcommand{\bc}{\begin{center}}
	\newcommand{\ec}{\end{center}}
\newcommand{\bsp}{\begin{sloppypar}}
	\newcommand{\esp}{\end{sloppypar}}
\newcommand{\mname}[1]{\mbox{\sf #1}}
\newcommand{\sB}{\mbox{$\cal B$}}
\newcommand{\sC}{\mbox{$\cal C$}}
\newcommand{\sF}{\mbox{$\cal F$}}
\newcommand{\sM}{\mbox{$\cal M$}}
\newcommand{\sP}{\mbox{$\cal P$}}
\newcommand{\sV}{\mbox{$\cal V$}}
\newcommand{\set}[1]{{\{ #1 \}}}
\newcommand{\Neg}{\neg}
\ifdefined \And
\renewcommand{\And}{\wedge}
\else
\newcommand{\And}{\wedge}
\fi
\newcommand{\Or}{\vee}
\newcommand{\Implies}{\Rightarrow}
\newcommand{\Iff}{\LeftRightarrow}
\newcommand{\Forall}{\forall}
\newcommand{\ForallApp}{\forall\,}
\newcommand{\Forsome}{\exists}
\newcommand{\ForsomeApp}{\exists\,}
\newcommand{\mdot}{\mathrel.}
\newcommand{\eps}{\epsilon}
\newcommand{\pnote}[1]{\langle \mbox{#1} \rangle}


\begin{document}

\begin{center}

{\large \textbf{COMPSCI/SFWRENG 2FA3}}\\[2mm]
{\large \textbf{Discrete Mathematics with Applications II}}\\[2mm]
{\large \textbf{Winter 2021}}\\[8mm]
{\huge \textbf{Assignment 9}}\\[6mm]
{\large \textbf{Dr.~William M. Farmer and Dr.~Mehrnoosh Askarpour}}\\[2mm]
{\large \textbf{McMaster University}}\\[6mm]
{\large Revised: March 28, 2021}

\end{center}

\medskip

Assignment 9 consists of two problems.  You must write your solutions
to the problems using LaTeX.

Please submit Assignment~9 as two files,
\texttt{Assignment\_9\_\emph{YourMacID}.tex} and
\texttt{Assignment\_9\_\emph{YourMacID}.pdf}, to the Assignment~9
folder on Avenue under Assessments/Assignments.
\texttt{\emph{YourMacID}} must be your personal MacID (written without
capitalization).  The \texttt{Assignment\_9\_\emph{YourMacID}.tex}
file is a copy of the LaTeX source file for this assignment
(\texttt{Assignment\_9.tex} found on Avenue under
Contents/Assignments) with your solution entered after each problem.
The \texttt{Assignment\_9\_\emph{YourMacID}.pdf} is the PDF output
produced by executing

\begin{itemize}

  \item[] \texttt{pdflatex Assignment\_9\_\emph{YourMacID}}

\end{itemize}

This assignment is due \textbf{Sunday, April 4, 2021 before
  midnight.}  You are allow to submit the assignment multiple times,
but only the last submission will be marked.  \textbf{Late submissions
  and files that are not named exactly as specified above will not be
  accepted!}  It is suggested that you submit your preliminary
\texttt{Assignment\_9\_\emph{YourMacID}.tex} and
\texttt{Assignment\_9\_\emph{YourMacID}.pdf} files well before the
deadline so that your mark is not zero if, e.g., your computer fails
at 11:50 PM on April 4.

\textbf{Although you are allowed to receive help from the
  instructional staff and other students, your submission must be your
  own work.  Copying will be treated as academic dishonesty! If any of
  the ideas used in your submission were obtained from other students
  or sources outside of the lectures and tutorials, you must
  acknowledge where or from whom these ideas were obtained.}

\newpage

\subsection*{Problems}

\be

  \item \textbf{[10 points]} Let $\Sigma = \set{a,b}$ and \[L = \set{x
    \in \Sigma^* \mid \#a(x) \text{ and } \#b(x) \text{ are both
      even}}.\] Construct a total Turing machine that accepts $L$.
    Present the TM using a transition table or diagram formally, and
    describe how it works informally.

  \textcolor{blue}{\textbf{Mohammad Omar Zahir, zahirm1, April 4, 2021}}
    
    $M = (Q,\Sigma,\Gamma,\vdash,\sqcup,\delta,S,t,r) \;
    \mname{where:}\\
    Q = \{S,q_1,q_2,q_3,t,r\}\\ 
    \Sigma = \{a,b\}\\
    \Gamma = \{a,b,\vdash,\sqcup,x\}\\
    $\\
  \begin{tabular}{| l | l | l | l | l | p{2.4cm} |}
        \hline
        \textbf{} & $\vdash$ & $a$ & $b$ & $x$ & $\sqcup$\\
        \hline
        $S$ & $(S,\: \vdash,\:R)$ & $(q_1,\: x,\:R)$ & $(q_2,\:x,\:R)$ & $(S,x,R)$ & $(t,\:-,\:-)$\\
        \hline
        $q_1$ & $-$ & $(q_3,\: x,\:L)$ & $(q_1,\:b,\:R)$ & $(q_1,x,R)$ & $(r,\:-,\:-)$\\
        \hline
        $q_2$ & $-$ & $(q_2,\: a,\:R)$ & $(q_3,\:x,\:L)$ & $(q_2,x,R)$ & $(r,\:-,\:-)$\\
        \hline
        $q_3$ & $(S,\vdash,R)$ & $(q_3,\: a,\:L)$ & $(q_3,\:b,\:L)$ & $(q_3,x,L)$ & $-$\\
        \hline
        $t$ & $(t,\vdash,R)$ & $(t,-,-)$ & $(t,-,-)$ & $(t,-,-)$ & $(t,-,-)$\\
        \hline
        $r$ & $(r,\vdash,R)$ & $(r,-,-)$ & $(r,-,-)$ & $(r,-,-)$ & $(r,-,-)
        $\\
        \hline
  \end{tabular}

\medskip

  We begin in the starting state $S$ where we will begin accepting letters. If we reach the end of the word, that would mean that we have satisfied the condition of the language, as both $a$ and $b$ are at an even number of 0 times. If for example we get an $a$ or a $b$ at $S$, we will be at an odd number of $a$ or $b$, and we will send each to either $q_1$ or $q_2$, with each representing an odd number of $a$ and $b$ respectively. From this state we will continue moving forward, ignoring any $b$ or $x$ we come across for $q1$, and any $a$ or $x$ that we come across for $q_2$. Once we get another $a$ or $b$ for $q_1$ and $q_2$ respectively, that would mean that we now have an even number of $a$ or $b$, meaning the condition of the word is satisfied until now, which would be represented by the $q_3$ state which represents an even number of $a$ or $b$. Alternatively, if we reach the end of the word $\sqcup$ from the either $q_1$ or $q_2$, that would mean that we have an uneven number of the corresponding letter, meaning the word cannot be accepted and will go to the rejection state $r$. From $q_3$ we will then move back on the tape until we reach the beginning of the word, $\vdash$, putting us in state $S$ so we begin the same process over. If we move right from $S$ and we come across any $x$, we can ignore them as they would represent a matched letter of $a$ or $b$ somewhere else in the word. Alternatively, from $S$ if we reach the end of the word $\sqcup$ only passing $x$'s, that would mean that every letter has been matched with another letter, occurring an even number of times. Thus we could simply move to the accepting state $t$ in that situation to accept the string. 

  \item \textbf{[10 points]} Let $\Sigma = \set{a,b}$ and \[L = \set{x
    \in \Sigma^* \mid \#a(x) = \#b(x)}.\] Construct a total Turing
    machine that accepts $L$.  Present the TM using a transition table
    or diagram formally, and describe how it works informally.

  \bigskip

  \textcolor{blue}{\textbf{Mohammad Omar Zahir, zahirm1, April 4, 2021}}

  $M = (Q,\Sigma,\Gamma,\vdash,\sqcup,\delta,S,t,r)\;
    \mname{where:}\\
    Q = \{S,q_1,q_2,q_3,t,r\}\\ 
    \Sigma = \{a,b\}\\
    \Gamma = \{a,b,\vdash,\sqcup,x\}\\
    $

  \begin{tabular}{| l | l | l | l | l | p{2.4cm} |}
        \hline
        \textbf{} & $\vdash$ & $a$ & $b$ & $x$ & $\sqcup$\\
        \hline
        $S$ & $(S,\: \vdash,\:R)$ & $(q_1,\: x,\:R)$ & $(q_2,\:x,\:R)$ & $(S,x,R)$ & $(t,\:-,\:-)$\\
        \hline
        $q_1$ & $-$ & $(q_1,\: a,\:R)$ & $(q_3,\:x,\:L)$ & $(q_1,x,R)$ & $(r,\:-,\:-)$\\
        \hline
        $q_2$ & $-$ & $(q_3,\: x,\:L)$ & $(q_2,\:b,\:R)$ & $(q_2,x,R)$ & $(r,\:-,\:-)$\\
        \hline
        $q_3$ & $(S,\vdash,R)$ & $(q_3,\: a,\:L)$ & $(q_3,\:b,\:L)$ & $(q_3,x,L)$ & $-$\\
        \hline
        $t$ & $(t,\vdash,R)$ & $(t,-,-)$ & $(t,-,-)$ & $(t,-,-)$ & $(t,-,-)$\\
        \hline
        $r$ & $(r,\vdash,R)$ & $(r,-,-)$ & $(r,-,-)$ & $(r,-,-)$ & $(r,-,-)
        $\\
        \hline
  \end{tabular}
  
    \medskip

  We begin in the start state $S$, where we can either read an $a$ or a $b$, which will then correspondingly go to the states $q_1$ and $q_2$, respectively. These states represent an unmatched $b$ for an $a$, and an unmatched $a$ for a $b$ respectively. From here we will continue moving forward until $q_1$ finds a $b$ towards the right, or $q_2$ finds an $a$ towards the right, to which it will go to $q_3$, which respresents a matched number of $a$'s and $b$'s. If we are in $q_1$ or $q_2$ and we do not find the corresponding letter and reach the end of the word, or $\sqcup$, that would imply that the $a$ does not have a matching $b$, or a $b$ does not have a matching $a$, meaning the number of $a$ and $b$ in the word is not the same. We would thus go from $q_1$ or $q_2$ in this position to the rejection state $r$. Otherwise, we then move backwards from $q_3$ until we reach the start symbol, staying at $q_3$ for anything else. Once at the start symbol, we will then begin the cycle described above again for if we get an $a$ or a $b$, or simply stay in $S$ if we get an $x$. The word is accepted, however, when we are back at the start state and we make it to the end of the word, or $\sqcup$, which would imply there are no more $a$'s or $b$'s, and we stayed at $S$ if there were $x$'s. Hence, that would mean that all the $a$'s and $b$'s have been matched thus far and if we reach the end of the input, we can go to the accepting state, $t$.

\ee

\end{document}
